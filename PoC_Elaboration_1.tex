\documentclass{article}
\usepackage[utf8]{inputenc}

\title{Proof of Concept Elaboration 1}
\author{Jesse Grubesich}
\date{August 2019}

\begin{document}

\maketitle

\section{Introduction}
For this elaboration, I will be focusing on the transcription and translation of Cyrillic and Latin script, and Croatian/Serbian/English language. I will be confirming whether or not this is of an appropriate scope in Friday's (30/8/2019) lesson. My concern is not that it will be too big, but that it will be to small; the word "small" is useless without a reference point, true, but I am unsure of the minimum size if there is one. If this does meet the requirements, then I am confident that this will be challenging enough to provide a learning experience and also possible within the bounds of my limited experience.

My project would be specialised for three languages and two scripts; therefore, the accuracy of transcriptions, translations, and possible translations is a primary concern. I would also attempt to provide indicators for human users as to whether a word is a noun, verb, adjective, what its case is (for Croatian and Serbian), what its role in the sentence is, possible synonyms and precise meanings, and so on. Essentially, I want to provide human users, including myself, with as much information as possible about these recommended translations so that they can make informed decisions about what they should change. However, I also want to give users the option of an accurate, automatic full-text translation for journal articles and such - both options are desirable to me. I have included this explanation in case there is any question as to whether the APIs would be able to solve my entire problem for me.

I have next to no experience with programming languages, software libraries, and APIs, but I have made some discoveries through my own research.

\section{APIs/Software Libraries}
\subsection{Google Translate}
I was unsure of the exact definition of API until I found a few examples of them and started to see their functions demonstrated to me. I have also seen that the terms API and software library are frequently conflated, so I am unsure as to which to use. Google translate is apparently one such API or software library.

Google Translate is the most obvious item to observe when considering my task. Google translate already does some of the things I would like my project to do; that is to say, it does the core thing - translation and transcription - albeit shoddily and unreliably. My project would be able to capitalise on this API's exiting functions while also building on them, as explained in the introduction.

\subsection{Similar APIs}
I have been lucky enough to find a wealth of APIs that could help me at the following page: https://rapidapi.com/collection/translation-apis. Some of the APIs here are jokes, but there are a few that stand out as being potentially useful to me:
\begin{itemize}
\item IBMWatsonLanguageTranslator, which automatically detects the language used in my input text (unless I misunderstand, and they mean programming language, in which case it may still be helpful, but for different reasons)
\item SYSTRAN.io, a collection of APIs for translation and multilingual dictionary lookups
\item MyMemory, Microsoft Text Translate, YandexTranslate, and Trans-lator all appear to be alternatives with close functions to Google Translate. They all appear to be free aside from Trans-lator. I would have to test these to find out which is the most suited for my task, but at least I have options.
\end{itemize}

All of these APIs cold be used to solve the first step - transcribe and translate.

\subsection{APIs for Grammar Analysis}
I also managed to find a number of APIs that could be used for my second step - analyse and report on grammar. APIs like TextAnalysis, LanguageTool, and Perfect Tense all have the functoinalities I'm looking for - they analyse grammar and flag problems.

Unfortunately, I so far haven't found any APIs that could help me with Croatian and Serbian grammar and Cyrillic-Latin transcription.


\section{Programming Languages}
I have very little idea on how to start ranking the efficacy of different programming languages in my project, as I have never programmed before or used any of these languages as far as I know. The common/popular ones appear to be Java, C, C++, Python, and Go. I gave a cursory glance to all of these, except for Python, on which I focused a little more, as it's part of a lesson in Data Carpentry, which I imagine indicates it's somewhat easier to learn than the others.

\subsection{Python}
Python is supposedly good for beginners, as it emphasises readability. As a beginner, I am much more interested in accessibility and readability than nuance in programming languages, so, without knowing any of its actual functions or how it works, Python seems like a good choice for my minimal experience.

\subsection{Low-Level Programming Languages}
When I searched these up, I imagined that they would be easier to use than high-level languages based on their name. It appears I was wrong. I briefly looked at Assembly and Machine Code and found that "low-level" means "micro-control", rather than "easy to use", so it is quite the opposite. While low-level programming languages may be helpful in my project, I doubt I will be using them due to my lack of experience, but I may re-evaluate this in the future. It is good to have such options, in any case.
\end{document}
